\pagenumbering{roman}
\setcounter{page}{1}
\pagestyle{plain}

%%%%%%%%%%%%%%%%
%%% COPYRIGHT %%%
%%%%%%%%%%%%%%%%
\chapter*{Copyright}

This work is licensed under a Creative Commons Attribution-NonCommercial-NoDerivatives 4.0 International License.

\vspace{1cm}

\begin{figure}[ht]
\centering
\includegraphics[scale=1]{images/license.png}
\end{figure}

Attribution-NonCommercial-NoDerivatives 4.0 International (CC BY-NC-ND 4.0)

\href{https://creativecommons.org/licenses/by-nc-nd/4.0/}{https://creativecommons.org/licenses/by-nc-nd/4.0/}

\vspace{1cm}

\textbf{Source Code:}

The complete source code for this project is publicly available at:

\url{https://github.com/begoechavarren/factible}

%%%%%%%%%%%%%
%%% RECORD %%%
%%%%%%%%%%%%%
\chapter*{FINAL PROJECT RECORD}

\begin{table}[ht]
\centering{}
\renewcommand{\arraystretch}{2}
\begin{tabular}{r | l}
\hline
Title of the project: & Factible: A Multi-Agent System for Automated\\
 & Fact-Checking of YouTube Videos\\
\hline
Author's name: & Bego\~na Echavarren S\'anchez\\
\hline
Tutor's name: & Josep-Anton Mir Tutusaus\\
\hline
Delivery date: & 12/2025\\
\hline
Degree or program: & Master's Degree in Data Science\\
\hline
Final Project area: & Natural Language Processing\\
\hline
Language of the project: & English\\
\hline
Keywords: & fact-checking, LLM, multi-agent systems\\
\hline
\end{tabular}
\end{table}

%%%%%%%%%%%%%%%%%%%
%%% ACKNOWLEDGEMENTS %%%
%%%%%%%%%%%%%%%%%%%
\chapter*{Acknowledgements}

I would like to thank my tutor, Josep-Anton Mir Tutusaus, for his support since the very beginning of this project. His warm reception of the idea, continuous encouragement, and genuine enthusiasm were truly appreciated. His expertise and insightful feedback definitely helped shape this work in a meaninful way.

I would also like to thank my family and friends for their patience, encouragement, and understanding---not only during these past months, but throughout my entire master's studies. Their continuous support has been essential in helping me develop the skills that made this thesis possible.

%%%%%%%%%%%%%%%%
%%% ABSTRACT %%%
%%%%%%%%%%%%%%%%
\chapter*{Abstract}
\addcontentsline{toc}{chapter}{Abstract}

\onehalfspacing

Misinformation on video platforms represents a growing challenge in the digital age. YouTube, with over 2 billion monthly users and 500 hours of video uploaded every minute, serves as a primary information source for millions worldwide. However, the absence of effective verification mechanisms allows false or misleading claims to spread at massive scale, impacting public health, democratic processes, and social cohesion.

This thesis presents Factible, a multi-agent system for automated fact-checking of YouTube videos. The system implements an end-to-end pipeline that processes video content through five specialized components: transcript extraction, claim detection with importance-based ranking, query generation for evidence retrieval, open-web evidence collection with source reliability assessment, and verdict synthesis with confidence levels.

The implementation leverages large language models (LLMs) for reasoning tasks while employing classical algorithms for deterministic operations. Key innovations include a novel approach for claim importance ranking, adaptive credibility filtering for source quality, and a three-level parallelization architecture for latency optimization.

Evaluation on 30 annotated YouTube videos across health, climate, and political topics demonstrates strong performance: 81.3\% claim extraction precision, 94.7\% evidence retrieval success rate, and 73.3\% verdict accuracy. The system achieves practical efficiency with mean processing time of 129.6 seconds per video at \$0.003 average cost, enabling cost-effective deployment for individual users.

The complete system, including a web-based interface with real-time streaming, is publicly available as open-source software, bridging the gap between academic research and practical fact-checking tools accessible to non-expert users.

\vspace{0.5cm}

\textbf{Keywords}: automated fact-checking, large language models, multi-agent systems, natural language processing, misinformation detection, YouTube, information retrieval, evidence-based verification
